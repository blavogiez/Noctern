\documentclass{article}
\usepackage[utf8]{inputenc}
\usepackage[french]{babel}

\begin{document}
% Inserted text begins here


\section{Introduction}
Ce document est conçu pour servir d'exemple de la création d'un fichier LaTeX compilable et simple. Il inclut les commandes essentielles nécessaires à l’impression standard d'une feuille, ainsi que des séquences éclaircies sur leur fonctionnement !

Nous verrons ici comment les commandes LaTeX sont utilisées pour structurer un document et créer des sections. Nous aborderons ensuite le processus de compilation pour assurer que notre travail s'affiche correctement sur papier ou dans une autre forme d'impression.

\subsection{Document Structure} 

Dans ce sous-section, nous explorerons les différentes parties d’un document LaTeX et comment elles s'assemblent pour former une structure cohérente. Nous commencerons par la définition des sections principales telles que l'introduction, le corps principal et la conclusion avant de passer aux subsections qui fournissent un cadre supplémentaire pour organiser les informations détaillées.

\end{document}