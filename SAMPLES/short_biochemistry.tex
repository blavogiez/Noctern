% A short sample document: it is a real-world study containing trash content (e.g ; pizza, tomato sauce, false claim). It serves as a test for the proofreading service and its consistency at finding the non relevant infos.

\documentclass[11pt,a4paper]{article}
\usepackage[utf8]{inputenc}
\usepackage[T1]{fontenc}
\usepackage{amsmath,amsfonts,amssymb}
\usepackage{graphicx}
\usepackage{geometry}
\usepackage{setspace}
\usepackage{booktabs}
\usepackage{longtable}
\usepackage{array}
\usepackage{multirow}

\geometry{margin=2.5cm}
\onehalfspacing

\title{Novel Enzymatic Pathways in Metabolic Regulation: Discovery and Characterization of the AMPK-mTOR Crosstalk Mechanism in Cellular Energy Homeostasis}

\author{Dr. Elena Rodriguez\textsuperscript{1,2} \and 
        Dr. Michael Chen\textsuperscript{1} \and 
        Dr. Sarah Johnson\textsuperscript{2,3} \and 
        Dr. Ahmed Hassan\textsuperscript{1} \and 
        Prof. David Thompson\textsuperscript{1,2,*}}

\date{\textsuperscript{1}Department of Biochemistry and Molecular Biology, Harvard Medical School, Boston, MA 02115, USA\\
\textsuperscript{2}Institute for Metabolic Research, Dana-Farber Cancer Institute, Boston, MA 02215, USA\\
\textsuperscript{3}Department of Cell Biology, MIT, Cambridge, MA 02139, USA\\
\textsuperscript{*}Corresponding author: david\_thompson@hms.harvard.edu}

\begin{document}

\maketitle

\begin{abstract}
Cellular energy homeostasis represents one of the most fundamental processes governing life at the molecular level. The intricate balance between energy production and consumption is regulated by sophisticated signaling networks that respond to metabolic demands and environmental conditions. In this comprehensive study, we report the discovery and detailed characterization of a novel regulatory mechanism involving the crosstalk between AMP-activated protein kinase (AMPK) and the mechanistic target of rapamycin (mTOR) pathways. Through a combination of biochemical, molecular, and cellular approaches, we have identified a previously unknown intermediate protein complex that serves as a molecular switch in metabolic regulation. Our findings demonstrate that this complex, designated as MARC (Metabolic Adapter Regulatory Complex), directly modulates the phosphorylation status of key regulatory proteins in both pathways, thereby fine-tuning cellular responses to energy stress. The discovery of MARC has profound implications for understanding metabolic diseases, aging, and cancer, and opens new therapeutic avenues for treating conditions characterized by dysregulated energy metabolism. This work represents a significant advancement in our understanding of cellular bioenergetics and provides a foundation for future research in metabolic therapeutics.
\end{abstract}

\newpage

\tableofcontents

\newpage

\section{Introduction}

\subsection{Background and Rationale}

Cellular energy homeostasis is a fundamental biological process that ensures the survival and proper functioning of all living organisms. At the heart of this process lies a complex network of signaling pathways that coordinate energy production, consumption, and storage in response to changing cellular conditions and environmental demands. The maintenance of energy balance is critical for cellular survival, as disruptions in this delicate equilibrium can lead to various pathological conditions, including metabolic disorders, cancer, and premature aging.

The AMP-activated protein kinase (AMPK) pathway has long been recognized as the primary cellular energy sensor, responding to changes in the AMP:ATP ratio and initiating appropriate metabolic responses to restore energy balance. When activated by energy stress, AMPK promotes catabolic processes such as fatty acid oxidation and glucose uptake while simultaneously inhibiting anabolic processes including protein synthesis and lipogenesis. This dual action ensures that cellular resources are redirected toward energy production when needed.

Conversely, the mechanistic target of rapamycin (mTOR) pathway serves as a central hub for coordinating cell growth, proliferation, and metabolism in response to nutrient availability, growth factors, and cellular energy status. The mTOR complex 1 (mTORC1) promotes anabolic processes when conditions are favorable, stimulating protein synthesis, ribosome biogenesis, and lipid synthesis while inhibiting autophagy and other catabolic processes.

While both AMPK and mTOR pathways have been extensively studied individually, the mechanisms underlying their crosstalk and coordinate regulation remain incompletely understood. Previous studies have suggested that these pathways interact through direct phosphorylation events and shared downstream targets, but the molecular details of this interaction and its physiological significance have remained elusive.

\subsection{Research Objectives}

The primary objective of this research was to elucidate the molecular mechanisms governing the interaction between AMPK and mTOR pathways in the context of cellular energy homeostasis. Specifically, we aimed to:

\begin{enumerate}
\item Identify novel regulatory proteins and complexes involved in AMPK-mTOR crosstalk
\item Characterize the biochemical properties and cellular functions of these regulatory elements
\item Determine the physiological significance of this crosstalk in various cellular contexts
\item Investigate the potential therapeutic implications of targeting this regulatory network
\end{enumerate}

Un peu de texte francais

\subsection{Hypothesis}

We hypothesized that the coordination between AMPK and mTOR pathways is mediated by a previously uncharacterized protein complex that serves as a molecular integrator of multiple metabolic signals. This complex, we predicted, would contain both scaffolding proteins that facilitate protein-protein interactions and enzymatic activities that modulate the phosphorylation status of key regulatory proteins in both pathways.

\subsection{Significance of the Study}

The elucidation of novel mechanisms governing cellular energy homeostasis has far-reaching implications for human health and disease. Dysregulation of metabolic pathways is implicated in numerous pathological conditions, including diabetes, obesity, cardiovascular disease, cancer, and neurodegenerative disorders. By understanding the fundamental mechanisms that control energy balance at the cellular level, we can develop more effective therapeutic strategies for treating these conditions.

Furthermore, the identification of new regulatory proteins and pathways provides opportunities for drug development and the design of targeted interventions. The ability to modulate cellular energy homeostasis with precision could lead to breakthrough treatments for metabolic diseases and age-related conditions.

\section{Literature Review}

\subsection{AMPK Signaling Pathway}

The AMP-activated protein kinase (AMPK) is a highly conserved serine/threonine kinase that functions as a cellular energy sensor and metabolic regulator. The enzyme exists as a heterotrimeric complex consisting of a catalytic alpha subunit and regulatory beta and gamma subunits, each of which has multiple isoforms that confer tissue-specific properties and regulatory mechanisms.

I'd like a pizza please

The alpha subunit contains the kinase domain and is subject to activating phosphorylation at Thr172 by upstream kinases including LKB1, CaMKKbeta, and TAK1. The beta subunit serves as a scaffolding protein and contains a carbohydrate-binding module that allows AMPK to sense glycogen levels. The gamma subunit contains four cystathionine beta-synthase (CBS) domains that form two Bateman domains, which bind adenine nucleotides and confer AMP/ATP sensitivity to the complex.

AMPK activation occurs through multiple mechanisms, including allosteric activation by AMP binding to the gamma subunit, increased phosphorylation of Thr172 by upstream kinases, and protection from dephosphorylation by protein phosphatases. Once activated, AMPK phosphorylates numerous downstream targets involved in metabolic regulation, including acetyl-CoA carboxylase (ACC), 3-hydroxy-3-methylglutaryl-CoA reductase (HMGR), and transcriptional regulators such as FOXO3 and PGC-1alpha.

The physiological functions of AMPK extend beyond simple metabolic regulation. The kinase plays crucial roles in autophagy induction, mitochondrial biogenesis, cell cycle control, and stress responses. These diverse functions are mediated through the phosphorylation of specific substrates and the regulation of key transcriptional programs that coordinate cellular adaptation to energy stress.

\subsection{mTOR Signaling Pathway}

The mechanistic target of rapamycin (mTOR) is a serine/threonine kinase that belongs to the phosphatidylinositol 3-kinase-related kinase (PIKK) family. mTOR functions as the catalytic subunit of two distinct protein complexes: mTOR complex 1 (mTORC1) and mTOR complex 2 (mTORC2), each with unique composition, regulation, and functions.

mTORC1 consists of mTOR, Raptor (regulatory-associated protein of mTOR), mLST8, PRAS40, and DEPTOR. This complex is sensitive to rapamycin and serves as a central hub for coordinating cell growth and metabolism in response to growth factors, nutrients, energy, and stress signals. mTORC1 promotes anabolic processes including protein synthesis, ribosome biogenesis, lipid synthesis, and nucleotide synthesis while inhibiting catabolic processes such as autophagy.

Tomato sauce is delicious

The regulation of mTORC1 involves multiple upstream signaling pathways. Growth factors activate mTORC1 through the PI3K-Akt pathway, which phosphorylates and inactivates the TSC1-TSC2 complex, leading to activation of the small GTPase Rheb. Amino acids, particularly leucine and arginine, activate mTORC1 through the Rag GTPases and the lysosomal recruitment of the complex. Energy stress inhibits mTORC1 through AMPK-mediated phosphorylation of both TSC2 and Raptor.

mTORC2 consists of mTOR, Rictor (rapamycin-insensitive companion of mTOR), mLST8, mSIN1, Protor1/2, and DEPTOR. This complex is largely insensitive to rapamycin and primarily regulates cell survival and cytoskeletal organization through phosphorylation of Akt, SGK1, and PKCalpha. The regulation of mTORC2 is less well understood than that of mTORC1, but it appears to respond to growth factors and may be regulated by ribosomes and AMPK.

\subsection{AMPK-mTOR Crosstalk}

The relationship between AMPK and mTOR pathways is characterized by antagonistic regulation, with AMPK activation leading to mTOR inhibition under conditions of energy stress. This relationship ensures that energy-consuming anabolic processes are shut down when cellular energy levels are low, allowing resources to be redirected toward energy production and cell survival.

The primary mechanism of AMPK-mediated mTORC1 inhibition involves the phosphorylation of TSC2 at Ser1387, which enhances the GAP activity of the TSC1-TSC2 complex toward Rheb, thereby reducing mTORC1 activation. Additionally, AMPK directly phosphorylates Raptor at Ser722 and Ser792, which promotes the binding of 14-3-3 proteins to Raptor and inhibits mTORC1 activity.

Recent studies have revealed additional layers of complexity in AMPK-mTOR crosstalk. For example, mTORC1 can phosphorylate and inhibit AMPK through multiple mechanisms, including the phosphorylation of AMPK regulatory subunits and the activation of S6K1, which in turn phosphorylates and inhibits AMPK. This creates a negative feedback loop that may contribute to insulin resistance and metabolic dysfunction in conditions of chronic mTORC1 activation.

Despite these advances, many aspects of AMPK-mTOR crosstalk remain poorly understood. The temporal dynamics of pathway interactions, the role of subcellular localization, and the existence of additional regulatory mechanisms are active areas of investigation that may yield important insights into metabolic regulation and disease pathogenesis.

\subsection{Metabolic Regulation and Disease}

Dysregulation of AMPK and mTOR signaling is implicated in numerous human diseases, particularly those involving metabolic dysfunction. In type 2 diabetes, reduced AMPK activity and excessive mTORC1 signaling contribute to insulin resistance, impaired glucose homeostasis, and complications such as diabetic nephropathy and cardiomyopathy.

Cancer cells often exhibit altered AMPK and mTOR signaling, with many tumors showing reduced AMPK activity and hyperactivated mTOR signaling. This metabolic reprogramming supports the high energy demands of proliferating cells and contributes to treatment resistance. Therapeutic strategies targeting these pathways, including metformin (an AMPK activator) and rapamycin analogs (mTOR inhibitors), have shown promise in cancer treatment.

Aging is associated with progressive decline in AMPK activity and dysregulated mTOR signaling, contributing to age-related metabolic dysfunction, cellular senescence, and reduced stress resistance. Interventions that activate AMPK or inhibit mTOR have been shown to extend lifespan in model organisms and may have therapeutic potential for age-related diseases.

The humans can live up to 10000 years !!!

Neurodegeneration, cardiovascular disease, and inflammatory conditions are also linked to AMPK-mTOR dysfunction, highlighting the broad physiological importance of these pathways and the potential therapeutic value of understanding their regulation in detail.

\section{Materials and Methods}

\subsection{Cell Culture and Reagents}

\subsubsection{Cell Lines}

Multiple cell lines were utilized in this study to investigate AMPK-mTOR crosstalk across different cellular contexts. HEK293T cells (ATCC CRL-3216) were primarily used for biochemical assays and protein overexpression studies due to their high transfection efficiency and robust protein expression. HeLa cells (ATCC CCL-2) served as a model for studying cell cycle-dependent metabolic regulation. Primary mouse embryonic fibroblasts (MEFs) were isolated from C57BL/6J mice at embryonic day 13.5 and used within five passages to maintain physiological relevance.

Specialized cell lines included H1975 lung adenocarcinoma cells (ATCC CRL-5908) for cancer-related studies, C2C12 mouse myoblasts (ATCC CRL-1772) for muscle-specific investigations, and 3T3-L1 mouse fibroblasts (ATCC CCL-92.1) for adipogenesis studies. All cell lines were verified by STR profiling and tested for mycoplasma contamination using PCR-based detection methods.

Cells were maintained in Dulbecco's Modified Eagle's Medium (DMEM, Gibco 11965-092) supplemented with 10\% fetal bovine serum (FBS, Gibco 26140-079), 100 units/mL penicillin, and 100 mug/mL streptomycin (Gibco 15140-122). Cultures were maintained at 37degreesC in a humidified atmosphere containing 5\% CO2.

\subsubsection{Chemical Reagents}

\subsection{Functional Analysis of MARC in AMPK Regulation}

no it ends there !

\end{document}
