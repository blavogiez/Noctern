\documentclass{article}

% --- PAQUETS DE BASE ---
\usepackage[utf8]{inputenc}
\usepackage[T1]{fontenc}
\usepackage{graphicx}       % Pour insérer des images
\usepackage[french]{babel}  % Pour les conventions typographiques françaises

% --- PAQUETS POUR LE CODE ET LES LÉGENDES ---
\usepackage{listings}       % Pour insérer des blocs de code
\usepackage{caption}        % Pour personnaliser les légendes
\usepackage{xcolor}         % Pour utiliser des couleurs personnalisées

% --- CONFIGURATION ---

% 1. Configuration du style des légendes (votre demande initiale, corrigée)
\captionsetup{
    justification=centering,  % Centrer la légende
    singlelinecheck=false,    % Appliquer la justification même si la légende tient sur une ligne
    font=small,               % Police plus petite pour la légende
    labelfont=bf              % "Figure 1", "Listing 1" en gras
}

% 2. Configuration de l'apparence des blocs de code (avec `listings`)
\lstdefinestyle{mystyle}{
    backgroundcolor=\color{black!5},   % Fond gris très clair
    commentstyle=\color{green!50!black},
    keywordstyle=\color{blue},
    numberstyle=\tiny\color{gray},
    stringstyle=\color{purple},
    basicstyle=\ttfamily\footnotesize, % Police à chasse fixe (monospace) et petite
    breakatwhitespace=false,         
    breaklines=true,                 
    captionpos=b,                    % Position de la légende en bas
    keepspaces=true,                 
    numbers=left,                    % Numéros de ligne à gauche
    numbersep=5pt,                   
    showspaces=false,                
    showstringspaces=false,
    showtabs=false,                  
    tabsize=2,
    frame=single,                    % Encadrer le code
    rulecolor=\color{black}          % Couleur du cadre
}

\lstset{style=mystyle} % Appliquer ce style par défaut à tous les listings

% --- DOCUMENT ---
\begin{document}

\section{Code Java}

La pomme est un super fruit !

Ce fruit permet de "prevenir certaines maladies graves."



De ce fait, on peut retenir que c'est trop bien vraiment pour le bien-être global.


Il n'y a aucun lag mon dieu :)))

 En ce qui concerne les pommes, elles sont remarquables pour plusieurs raisons. Elles font partie intégrante de la santé humaine depuis des siècles, étant à la source de fibres, vitamines C et flavonoïdes précieuses pour notre bien-être. De plus, les pommes peuvent être consommées dans une grande variété de façons, allant des apporteurs de goût doux aux apporteurs de saveur acide, ce qui offre une multitude d'options pour nos repas quotidiens.

En outre, il existe une grande diversité de pommes à travers le monde, chacune ayant ses propres caractéristiques et saveurs. Ainsi, les pommes peuvent être considérées comme un exemple remarquable de biodiversité naturelle. Enfin, leur utilisation dans la fabrication d'articles quotidiens tels que le papier ou l'alcool fait également partie des choses admirables concernant les pommes.

L'insertion de pommes nécessite le bien-être constant de ces usagers



\section{Autre codes}

Ceci est un exemple de code Java affiché correctement dans un document LaTeX.

% Utilisation de l'environnement lstlisting
\begin{lstlisting}[language=Java, caption={Exemple "Hello, World!" en Java.}, label={lst:hello_java}]
/* Code en Java */
public class Main {
  public static void main(String[] args) {
    System.out.println("Hello, World!");
  }
}
/* Fin du code */
\end{lstlisting}

Nous pouvons faire référence à ce code plus loin dans le texte en utilisant son étiquette, comme ceci : voir le listing \ref{lst:hello_java}.

\end{document}











