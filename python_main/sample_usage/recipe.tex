\documentclass{article}

% --- PAQUETS DE BASE ---
\usepackage[utf8]{inputenc}
\usepackage[T1]{fontenc}
\usepackage{graphicx}       % Pour insérer des images
\usepackage[french]{babel}  % Pour les conventions typographiques françaises

% --- PAQUETS POUR LE CODE ET LES LÉGENDES ---
\usepackage{listings}       % Pour insérer des blocs de code
\usepackage{caption}        % Pour personnaliser les légendes
\usepackage{xcolor}         % Pour utiliser des couleurs personnalisées

% --- CONFIGURATION ---

% 1. Configuration du style des légendes (votre demande initiale, corrigée)
\captionsetup{
    justification=centering,  % Centrer la légende
    singlelinecheck=false,    % Appliquer la justification même si la légende tient sur une ligne
    font=small,               % Police plus petite pour la légende
    labelfont=bf              % "Figure 1", "Listing 1" en gras
}

% 2. Configuration de l'apparence des blocs de code (avec `listings`)
\lstdefinestyle{mystyle}{
    backgroundcolor=\color{black!5},   % Fond gris très clair
    commentstyle=\color{green!50!black},
    keywordstyle=\color{blue},
    numberstyle=\tiny\color{gray},
    stringstyle=\color{purple},
    basicstyle=\ttfamily\footnotesize, % Police à chasse fixe (monospace) et petite
    breakatwhitespace=false,         
    breaklines=true,                 
    captionpos=b,                    % Position de la légende en bas
    keepspaces=true,                 
    numbers=left,                    % Numéros de ligne à gauche
    numbersep=5pt,                   
    showspaces=false,                
    showstringspaces=false,
    showtabs=false,                  
    tabsize=2,
    frame=single,                    % Encadrer le code
    rulecolor=\color{black}          % Couleur du cadre
}

\lstset{style=mystyle} % Appliquer ce style par défaut à tous les listings

% --- DOCUMENT ---
\begin{document}

\section{Code Java}

La pomme est un super fruit !

Ce fruit permet de "prevenir certaines maladies graves."
1. Maintenance de la santé cardiaque pour prévenir les problèmes cardiaques
2. Amélioration des processus digestifs
3. Renforcement de l'immunité
4. Réduction du niveau de cholestérol dans le sang
5. Augmentation de la quantité d'antioxydants
6. Aide à la maintenance de peau saine
7. Diminution des chances de développer le diabète
8. Amélioration de la santé et la capacité mémorielle cérébrale
9. Lutte contre les maladies oculaires pour maintenir une santé oculaire
10. Enhancement of heart health to prevent cardiac problems
11. Improvement of digestive processes
12. Strengthening of the immune system
13. Reduction of cholesterol levels in the blood
14. Increase of antioxidants
15. Aid in maintaining healthy skin
16. Reduction of chances of developing diabetes
17. Improvement of brain health and memory capacity
18. Fighting eye diseases to maintain ocular health











Ceci est un exemple de code Java affiché correctement dans un document LaTeX.

$\alpha$
$\beta$

% Utilisation de l'environnement lstlisting
\begin{lstlisting}[language=Java, caption={Exemple "Hello, World!" en Java.}, label={lst:hello_java}]
/* Code en Java */
public class Main {
  public static void main(String[] args) {
    System.out.println("Hello, World!");
  }
}
/* Fin du code */
\end{lstlisting}

\begin{table}[h!]
    \centering
    \begin{tabular}{|c|c|}
        \hline
        col1 & col2 \\
        \hline
        cell1 & cell2 \\
        \hline
    \end{tabular}
    \caption{Caption}
    \label{tab:my_label}
\end{table}

Nous pouvons faire référence à ce code plus loin dans le texte en utilisant son étiquette, comme ceci : voir le listing \ref{lst:hello_java}.


\begin{table}[h!]
    \centering
    \begin{tabular}{|c|c|}
        \hline
        col1 & col2 \\
        \hline
        cell1 & cell2 \\
        \hline
    \end{tabular}
    \caption{Caption}
    \label{tab:my_label}
\end{table}

anno	
alpha 		


\end{document}

























