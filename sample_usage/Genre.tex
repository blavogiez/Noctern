\documentclass[12pt, a4paper]{article}
   \usepackage[utf8]{inputenc}
   \usepackage[T1]{fontenc}
   \usepackage[frenchb]{babel}
   \usepackage{amsmath, amssymb, amsthm}
   \usepackage{graphicx}
   \usepackage{hyperref}
   \usepackage[top=3cm, bottom=2cm, left=3cm, right=3cm]{geometry}
   \title{Biographie d'Auguste Renoir}
   \author{}
   \date{}
   \begin{document}
   \maketitle
   \section*{Introduction}
   \noindent
   Auguste Renoir est né le 25 février 1841 à Limoges, dans le département de la Haute-Vienne. Il est considéré comme l'un des plus grands peintres impressionnistes français, dont les œuvres ont eu une grande influence sur l'art moderne.
   \section*{Enfance et formation}
   \noindent
   Après avoir été apprenti dans plusieurs ateliers de porcelaine à Limoges, Auguste Renoir a rejoint l'École des Beaux-Arts de Fontainebleau en 1862. Il a ensuite fréquenté l'Académie Julian et la Grande Chaumière à Paris.
   \section*{Carrière artistique}
   \noindent
   Auguste Renoir est connu pour ses peintures à thème figuratif, qui ont souvent été critiquées par les membres de l'École des Beaux-Arts pour leur liberté de composition et leur utilisation de couleurs vives. Cependant, son talent a rapidement été reconnu et il est devenu un artiste à part entière en 1864, lorsqu'il a exposé ses premières œuvres au Salon des Refusés.
   \section*{Influences et styles}
   \noindent
   Auguste Renoir a été influencé par les peintres impressionnistes tels que Claude Monet, Pierre-Auguste Renoir et Camille Pissarro. Son style est caractérisé par l'utilisation de couleurs vives, de compositions informelles et d'une grande liberté de traitement des sujets. Il est également connu pour son amour du nu féminin, qu'il a représenté dans de nombreuses œuvres.
   \section*{Vie personnelle}
   \noindent
   Auguste Renoir s'est marié deux fois et a eu plusieurs enfants. Il a également été le père adoptif d'un enfant. Sa dernière épouse, Aline Charigot, est morte en 1915. Après sa mort, Auguste Renoir a continué à peindre jusqu'à son décès en 1919.
   \section*{Fin de vie et postérité}
   \noindent
   Auguste Renoir est mort le 3 décembre 1919 à Cagnes-sur-Mer, dans les Alpes-Maritimes. Il a été enterré au cimetière de Belle-Île-en-Mer, en Morbihan. Son travail continue d'être exposé et apprécié dans le monde entier, et il est considéré comme l'un des grands maîtres de la peinture française.

   \end{document}
