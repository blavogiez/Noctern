\documentclass[12pt, a4paper]{article}
  \usepackage[utf8]{inputenc}
  \usepackage[T1]{fontenc}
  \usepackage[french]{babel}
  \usepackage{graphicx}

  \title{La Ville de Mexico}
  \author{}
  \date{}

  \begin{document}

   \maketitle

\section*{Mexico City}

   La ville de Mexico, capitale du Mexique, est une métropole vibrante et cosmopolite où s'entremêlent l'ancien et le moderne. Elle compte plus de 21 millions d'habitants, ce qui la place parmi les villes les plus peuplées au monde. Sa richesse historique et culturelle est exceptionnelle : ruines aztèques, temples aztecs, palais coloniaux, monuments modernes, musées riches de trésors artistiques, etc. La gastronomie mexicaine, riche et variée, offre une expérience culinaire inoubliable aux visiteurs.

	La diversité culturelle et la dynamique urbaine de Mexico City se manifestent également dans ses nombreux festivals d'anniversaire qui intègrent des traditions ancestrales avec l'art moderne. Le Festival Internacional Cervantino, par exemple, combine spectacles de danse contemporaine et performances théâtrales traditionnelles pour célébrer la poète Miguel de Cervantes. Les festivals d'anniversaire en ville sont une vraie fête des sens qui mettent en lumière le patrimoine vivant ainsi que les tendances artistiques actuelles du Mexique."

Mexico City prides itself on this blend during its festivals d’anniversaire. The city becomes a kaleidoscope of colors, sounds, and tastes as the population revels in these celebrations that honor both history and modernity. From La Feria de Chapultepec to the Guelaguetza festival at Xochimilco'de los Jazmines y Flor del Monte, each event showcases a unique aspect of Mexico’s cultural identity through traditional music, dance, food, and crafts that have been passed down for generations. These festivals are not just about entertainment; they serve as vital threads in the social fabric, fostering community spirit and pride among residents and visitors alike while reinforcing a collective memory of Mexico’s rich past amidst its dynamic present.

Here is the stylized text:

As dusk falls upon these festivities, \textbf{streetlights} flicker on in synchrony with a vibrant ensemble of mariachi bands. The air is filled with the entrancing melodies that resonate from every corner as people begin to dance freely beneath them, their laughter mixing with music and merriment under the twinkling \textbf{sky}. In Mexico City's energetic atmosphere during these anniversaries celebrations, there seems to be a perpetual motion of joyfulness - locals and visitors alike move in unison as they honor traditions that are intrinsic to their identity. The festivals not only serve as platforms for cultural expression but also actively promote inclusivity among the diverse communities inhabiting this metropolis.

Note: I have used the \textbf{extbf} command to boldify the words "streetlights" and "sky".

  \end{document}









