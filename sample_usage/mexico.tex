\documentclass[12pt, a4paper]{article}
\usepackage[utf8]{inputenc}
\usepackage[T1]{fontenc}
\usepackage[french]{babel}
\usepackage{graphicx}

\title{La \textit{Ville de Mexico}}
\author{}
\date{}

\begin{document}

\maketitle

\section*{\textbf{Mexico City}}

\textbf{Mexico City}, capitale du \textit{Mexique}, est une métropole dynamique et fascinante. Son histoire millénaire, son architecture impressionnante et sa scène culturelle vibrante en font une destination de choix.

\subsection*{\textbf{Histoire et Culture}}

Fondée par les \textit{Aztèques} au \textbf{XIVe siècle} sous le nom de \textit{Tenochtitlan}, la ville a été le théâtre d'événements historiques majeurs, de la conquête \textit{espagnole} à la révolution \textit{mexicaine}. Aujourd'hui, elle offre une richesse culturelle sans égale avec ses musées, ses galeries d'art et ses traditions vivantes.

\subsection*{\textbf{Architecture}}


\begin{figure}[h!]
    \centering
    \includegraphics[width=0.8\textwidth]{figures/textbfmexico_city/textbfarchitecture/default/fig_1.png}
    \caption{Caption here}
    \label{fig:textbfmexico_city_textbfarchitecture_1}
\end{figure}

\merguez


\textbf{Mexico City} présente une diversité architecturale remarquable, allant des vestiges préhispaniques aux édifices coloniaux et aux gratte-ciel modernes. Le centre historique, classé au patrimoine mondial de l'\textit{UNESCO}, abrite des monuments emblématiques tels que le \textit{Zócalo} et la cathédrale métropolitaine.

\subsection*{\textbf{Gastronomie}}

La scène gastronomique de \textbf{Mexico City} est réputée dans le monde entier. Des stands de rue proposant des \textit{tacos} et des \textit{tamales} aux restaurants étoilés \textit{Michelin}, la ville offre une palette de saveurs exquises qui reflètent la richesse de la cuisine \textit{mexicaine}.

\subsection*{\textbf{Sites d'intérêt}}

Parmi les nombreux sites à visiter, citons le \textit{Musée National d'Anthropologie}, le \textit{Palais des Beaux-Arts}, le quartier de \textit{Coyoacán} avec la maison de \textit{Frida Kahlo}, et le site archéologique de \textit{Teotihuacan}, facilement accessible depuis la ville.


\end{document}

