\documentclass[12pt, a4paper]{article}
\usepackage[utf8]{inputenc}
\usepackage[T1]{fontenc}
\usepackage[french]{babel}
\usepackage{graphicx}

\title{La \textit{Ville de Mexico}}
\author{}
\date{}

\begin{document}

\maketitle

\section*{\textbf{Mexico City}}

Le cœur vibrant du \textit{Mexique} bat dans sa capitale tentaculaire, une ville qui marie harmonieusement l'histoire ancienne à un esprit moderne dynamique.

\textbf{Mexico City}, capitale du \textit{Mexique}, est une métropole dynamique et fascinante. Son histoire millénaire, son architecture impressionnante et sa scène culturelle vibrante en font une destination de choix.

\subsection*{\textbf{Histoire et Culture}}

Fondée par les \textit{Aztèques} au \textbf{XIVe siècle} sous le nom de \textit{Tenochtitlan}, la ville a été le théâtre d'événements historiques majeurs, de la conquête \textit{espagnole} à la révolution \textit{mexicaine}. Aujourd'hui, elle offre une richesse culturelle sans égale avec ses musées, ses galeries d'art et ses traditions vivantes.

\subsection*{\textbf{Architecture}}

La ligne où je suis compte p

```latex

\begin{figure}[h!]
    \centering
    \begin{tabular}{|c|c|c|c|c|c|c|}
        \hline
        \textbf{\'{E}tat} & \textbf{Capitale} & \textbf{Population (approx.)} & \textbf{Superficie (km²)} & \textbf{Code \texttt{INEGI}} & \textbf{Gentil\'{e}} & \textbf{Principale activit\'{e} \'{e}conomique} \\
        \hline
        \textit{Aguascalientes} & \textit{Aguascalientes} & 1,4 millions & 5,615 & \texttt{AGU} & Hidroc\'{a}lido(a) & Industrie automobile \\
        \textit{Baja California} & \textit{Mexicali} & 3,7 millions & 71,446 & \texttt{BCN} & Bajacaliforniano(a) & Tourisme, industrie manufacturière \\
        \textit{Baja California Sur} & \textit{La Paz} & 0,8 millions & 73,909 & \texttt{BCS} & Sudcaliforniano(a) & Tourisme \\
        \textit{Campeche} & \textit{San Francisco de Campeche} & 0,9 millions & 57,924 & \texttt{CAM} & Campechano(a) & P\'{e}trole et gaz, tourisme \\
        \textit{Chiapas} & \textit{Tuxtla Guti\'{e}rrez} & 5,7 millions & 73,211 & \texttt{CHP} & Chiapaneco(a) & Agriculture, tourisme \\
        \textit{Chihuahua} & \textit{Chihuahua} & 3,7 millions & 247,087 & \texttt{CHH} & Chihuahuense & Industrie manufacturière, agriculture \\
        \textit{Ciudad de M\'{e}xico} & (Aucune) & 9,2 millions & 1,495 & \texttt{CDMX} & Chilango(a) & Services, commerce, tourisme \\
        \hline
    \end{tabular}
    \caption{Aper\c{c}u des \'{E}tats Mexicains}
    \label{tab:etats_mexicains}
\end{figure}
    
% it is a comment
\begin{table}[h!]
    \centering
    \begin{tabular}{|c|c|}
        \hline
        mcdonalds & burgerking \\
        \hline
        bigmac & bigburgerking \\
        \hline
potatoes & idk \\
        \hline
    \end{tabular}
    \caption{Mcdonals vs burgerking}
    \label{tab:my_label}
\end{table}

\textbf{Mexico City} présente une diversité architecturale remarquable, allant des vestiges préhispaniques aux édifices coloniaux et aux gratte-ciel modernes. Le centre historique, classé au patrimoine mondial de l'\textit{UNESCO}, abrite des monuments emblématiques tels que le \textit{Zócalo} et la cathédrale métropolitaine.

\subsection*{\textbf{Gastronomie}}

La scène gastronomique de \textbf{Mexico City} est réputée dans le monde entier. Des stands de rue proposant des \textit{tacos} et des \textit{tamales} aux restaurants étoilés \textit{Michelin}, la ville offre une palette de saveurs exquises qui reflètent la richesse de la cuisine \textit{mexicaine}.

\subsection*{\textbf{Sites d'intérêt}}

Parmi les nombreux sites à visiter, citons le \textit{Musée National d'Anthropologie}, le \textit{Palais des Beaux-Arts}, le quartier de \textit{Coyoacán} avec la maison de \textit{Frida Kahlo}, et le site archéologique de \textit{Teotihuacan}, facilement accessible depuis la ville.
\end{document}













