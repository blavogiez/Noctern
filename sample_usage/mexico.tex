\documentclass[12pt, a4paper]{article}
\usepackage[utf8]{inputenc}
\usepackage[T1]{fontenc}
\usepackage[french]{babel}
\usepackage{graphicx}

\title{La Ville de Mexico}
\author{}
\date{}

\begin{document}

\maketitle

\section*{Mexico City}

\section{Histoire de Mexico}
\subsection{Fondation de Tenochtitlan}
\section{Géographie de Mexico}
\subsection{Altitude et relief}
\section{Climat de Mexico}
\subsection{Variations saisonnières}
\section{Démographie de Mexico}
\subsection{Croissance démographique}
\section{Culture de Mexico}
\subsection{Traditions et coutumes}
\section{Gastronomie mexicaine}
\subsection{Plats typiques}
\section{Architecture de Mexico}
\subsection{Styles architecturaux}
\section{Musées de Mexico}
\subsection{Collections et expositions}
\section{Sites historiques de Mexico}
\subsection{Vestiges archéologiques}
\section{Vie nocturne de Mexico}
\subsection{Divertissement et loisirs}
\section{Économie de Mexico}
\subsection{Secteurs économiques}
\section{Transports à Mexico}
\subsection{Réseau de transport}
\section{Environnement de Mexico}
\subsection{Défis environnementaux}
\section{Politique de Mexico}
\subsection{Gouvernement local}
\section{Éducation à Mexico}
\subsection{Institutions éducatives}
\section{Art et culture populaire à Mexico}
\subsection{Expressions artistiques}
\section{Sécurité à Mexico}
\subsection{Conseils aux visiteurs}
\section{Tourisme à Mexico}
\subsection{Sites touristiques majeurs}
\section{Zones périphériques de Mexico}
\subsection{Extension urbaine}
\textit{aieie}


\begin{figure}[h!]
    \centering
    \includegraphics[width=0.8\textwidth]{figures/zones_périphériques_de_mexico/extension_urbaine/default/fig_1.png}
    \caption{Caption here}
    \label{fig:zones_périphériques_de_mexico_extension_urbaine_1}
\end{figure}




\end{document}




















































































