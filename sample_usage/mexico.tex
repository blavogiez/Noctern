\documentclass[12pt, a4paper]{article}
  \usepackage[utf8]{inputenc}
  \usepackage[T1]{fontenc}
  \usepackage[french]{babel}
  \usepackage{graphicx}

  \title{La Ville de Mexico}
  \author{}
  \date{}

  \begin{document}

   \maketitle

\textbf{La ville de Mexico}, ou \textit{Mexico D.F.} selon son nom complet, est la capitale de l'état de Mexico et également la ville la plus peuplée du Mexique. Située à environ 2 240 mètres d'altitude dans la vallée des Mejicas, elle est entourée par les \texttt{cordillères occidentales} à l'ouest et au sud.

La capitale mexicaine est célèbre pour ses monuments historiques et culturels tels que \textbf{\textit{le Zocalo}} (place principale), \textbf{\textit{le Palais des Beaux-Arts}}, \textbf{\textit{la Cathédrale métropolitaine}}, \textbf{\textit{ou encore les Pyramides de Tultepec}}. Elle possède également un patrimoine architectural remarquable comme \textbf{\textit{l'édifice du Palais des Gouverneurs}} et le \textbf{\textit{Monument à la Révolution mexicaine}}.

Depuis 1985, Mexico est classée par l'UNESCO comme une ville mondiale de l'artisanat populaire. La gastronomie mexicaine y est également très appréciée, avec des spécialités telles que les \textbf{\textit{tacos}}, les \textbf{\textit{enchiladas}} et la \textbf{\textit{mole}}.

Bien qu'affectée par la pollution, Mexico possède une vie nocturne active avec ses nombreux bars et discothèques. Elle est également connue pour son dynamisme culturel, économique et politique.



  \end{document}
