\documentclass[12pt, a4paper]{article}
\usepackage[utf8]{inputenc}
\usepackage[T1]{fontenc}
\usepackage[french]{babel}
\usepackage{graphicx}

\title{La Ville de Mexico}
\author{}
\date{}

\begin{document}

\maketitle

\section*{Mexico City}

\section{Histoire de Mexico}
\subsection{Fondation de Tenochtitlan}
\section{Géographie de Mexico}
\subsection{Altitude et relief}
\section{Climat de Mexico}
\subsection{Variations saisonnières}
\section{Démographie de Mexico}
\subsection{Croissance démographique}
\section{Culture de Mexico}
\subsection{Traditions et coutumes}
\section{Gastronomie mexicaine}
\subsection{Plats typiques}
\section{Architecture de Mexico}
\subsection{Styles architecturaux}
\section{Musées de Mexico}
\subsection{Collections et expositions}
\section{Sites historiques de Mexico}
\subsection{Vestiges archéologiques}
\section{Vie nocturne de Mexico}
\subsection{Divertissement et loisirs}
\section{Économie de Mexico}
\subsection{Secteurs économiques}
\section{Transports à Mexico}
\subsection{Réseau de transport}
\section{Environnement de Mexico}
\subsection{Gouvernement local}
\section{Éducation à Mexico}
\subsection{Institutions éducatives
\section{Art et culture populaire à Mexico}
\subsection{Expressions artistiques}

Conséquemment, Subsequently, urban development has frequently resulted in the conservation of historical parks and green spaces as ecological counterbalances within expanding cities. Initiatives aimed at incorporating these elements into new neighborhood projects underscore respect for both natural and cultural heritage while providing city dwellers with requisite leisure space.


\begin{figure}[h!]
    \centering
    \includegraphics[width=0.8\textwidth]{figures/zones_périphériques_de_mexico/extension_urbaine/default/fig_1.png}
    \caption{Caption here}
    \label{fig:zones_périphériques_de_mexico_extension_urbaine_1}
\end{figure}

In this photograph labeled as "fig:zones_périphériques_de_mexico_extension_urbaine_1," one can observe the integration of historical parks within urban expansion projects in peripheral zones, showcasing how Mexico's cities attempt to balance modern growth with heritage preservation.

This way, this integration serves as a model for sustainable urban development that prioritizes green space alongside residential and commercial areas. It demonstrates Mexico City’s efforts to address both the need for expansion due to population growth and the preservation of its cultural landmarks, thereby offering residents an oasis amidst concrete jungles while maintaining a connection with their historical identity through these preserved spaces.







\end{document}















































































































