\documentclass[12pt, a4paper]{article}
  \usepackage[utf8]{inputenc}
  \usepackage[T1]{fontenc}
  \usepackage[french]{babel}
  \usepackage{graphicx}

  \title{La Ville de Mexico}
  \author{}
  \date{}

  \begin{document}

   \maketitle

\textbf{La ville de Mexico}, ou \textit{Mexico D.F.} selon son nom complet, est la capitale de l'état de Mexico et également la ville la plus peuplée du Mexique. Située à environ 2 240 mètres d'altitude dans la vallée des Mejicas, elle est entourée par les \texttt{cordillères occidentales} à l'ouest et au sud.

La capitale mexicaine est célèbre pour ses monuments historiques et culturels tels que \textbf{\textit{le Zocalo}} (place principale), \textbf{\textit{le Palais des Beaux-Arts}}, \textbf{\textit{la Cathédrale métropolitaine}}, \textbf{\textit{ou encore les Pyramides de Tultepec}}. Elle possède également un patrimoine architectural remarquable comme \textbf{\textit{l'édifice du Palais des Gouverneurs}} et le \textbf{\textit{Monument à la Révolution mexicaine}}.

Since UNESCO recognized Mexico as a world city distinguished by popular craftsmanship in 1985, it has carved out a reputation within the culinary domain. Its cuisine is particularly esteemed for signature dishes such as \textbf{\textit{tacos}}, \textbf{\textit{enchiladas}} and richly flavored \textbf{\textit{mole}}.
Concurrent with ecological challenges, Mexico boasts a vibrant nightlife featuring various bars and discotheques. Furthermore, the nation is praised for its lively cultural, economic, and political life.S

Sustainability efforts have led to significant improvements in Mexico City's air quality over recent years. Initiatives such as expanding public transportation networks with an emphasis on electric buses are gradually transforming the urban mobility landscape, aiming to reduce reliance on personal vehicles and thus decrease pollution levels." Instruction 2 (More difficult by adding at least {5} more constraints):

	
  \end{document}


