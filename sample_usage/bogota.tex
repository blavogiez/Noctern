\documentclass{article}
\usepackage[utf8]{inputenc}
\usepackage[T1]{fontenc}
\usepackage{babel}
\usepackage{graphicx}
\usepackage{hyperref}
\usepackage{geometry}
\geometry{a4paper, margin=1in}

\title{Bogotá: An Overview of the Colombian Capital}
\author{A citizen!}
\date{\today}

\begin{document}

\maketitle

\section{A real story}

\section{A story of its kind}

\section{ifosoifdi}

\begin{itemize}
    \item premer
    \item deuxio
    \item tresio
    \item quatero
\end{itemize}

\section{A famous story}

DIide zid zi ize
fifezifezi
dsdqs
d
zz
About Colombia
Bogotá, officially named Bogotá, Capital District (in Spanish: Bogotá, Distrito Capital), is the capital and largest city of Colombia. It also serves as the capital of the department of Cundinamarca, though it is not an integral part of it. Located in the center of Colombia, on the Cundiboyacense high plateau, a high plain of the Eastern Cordillera of the Andes, Bogotá is a vibrant and multicultural metropolis, the econ
omic, cultural, and political crossroads of the country. With an estimated population of over 8 million inhabitants within the city limits and over 11 million in its metropolitan area, it is one of the largest cities in Latin America.

\section{Geography and Climate}
Bogotá's altitude is approximately 2,640 meters above sea level, which gives it a temperate mountain climate, characterized by relatively constant temperatures throughout the year. The average annual temperature is around 14.5 $^\circ$C, with daily variations being more pronounced than seasonal variations. The city experiences two dry seasons (December-February and July-August) and two wet seasons (March-May and September-November). Rainfall is frequent, often in the form of intense showers.

\subsection{Introduction}
Bogotá's position on the high plateau has historically favored its development. Surrounded by imposing mountains, including Monserrate and Guadalupe which overlook the city, it offers spectacular landscapes. The Bogotá River flows through the region and contributes to its agricultural richness, although it also faces significant environmental challenges.

\section{History}
Bogotá was founded on August 6, 1538, by the Spanish conquistador Gonzalo Jiménez de Quesada, who named it Santa Fe. The city quickly became an important administrative center under Spanish rule, becoming the seat of the Viceroyalty of New Granada in 1717.

Later, it played a crucial role in the independence movements of South America."

\subsection{Independence and Development}
Bogotá played a crucial role in Colombia's struggle for independence. It was the scene of several key events, notably the "Grito de Independencia" (Cry of Independence) on July 20, 1810. After independence, the city experienced progressive growth, becoming the political and economic heart of the nation. The 20th century saw rapid urbanization and a major transformation of the city into a modern metropolis.

\section{Economy}
Bogotá is the main economic and financial center of Colombia. It concentrates a significant portion of the national GDP and is home to the headquarters of numerous national and international companies, as well as major financial institutions. Key sectors of Bogotá's economy include services (finance, telecommunications, tourism), industry (automotive, textile, agribusiness), and commerce. The city is also a hub for innovation and technology.

\section{Culture and Tourism}
Bogotá offers remarkable cultural and historical richness, with numerous museums, theaters, art galleries, and monuments.

\subsection{Museums and Heritage}
Among the most famous cultural institutions are:
\begin{itemize}
    \item The Gold Museum (Museo del Oro): housing an exceptional collection of pre-Columbian gold artifacts.
    \item The Botero Museum (Museo Botero): dedicated to the work of Fernando Botero, a world-renowned Colombian artist.
    \item The National Museum of Colombia (Museo Nacional de Colombia): showcasing the country's history and art.
    \item La Quinta de Bolívar: the former residence of the liberator Simón Bolívar.
\end{itemize}

\subsection{Iconic Neighborhoods}
The city is divided into several areas, each with its own character:
\begin{itemize}
    \item La Candelaria: the historic district, with its cobblestone streets, colorful colonial houses, and bohemian atmosphere. It is the historical and cultural heart of the city.    
    \item Usaquén: a charming neighborhood to the north, formerly a village, known for its Sunday artisan market, restaurants, and preserved colonial architecture.
    \item Zona G (Gastronomic Zone) and Zona T (Leisure Zone): modern and lively areas, popular for their nightlife, restaurants, and shops.
\end{itemize}

\subsection{Tourist Attractions}
In addition to its museums and neighborhoods, Bogotá offers several attractions:
\begin{itemize}
    \item Monserrate: accessible by cable car or funicular, this peak offers spectacular panoramic views of the city.
    \item Plaza de Bolívar: the main square of the city, surrounded by important governmental and religious buildings such as the Primada Cathedral, the National Capitol, and the Palace of Justice.
    \item José Celestino Mutis Botanical Garden: an important green space dedicated to Colombian flora.
\end{itemize}

\section{Transportation}
Bogotá has a developed public transportation network, although it faces congestion challenges. The TransMilenio bus system, a Bus Rapid Transit (BRT) system, is the primary mode of public transport. The city is also working to improve its network of cycle paths, which is one of the longest in Latin America. El Dorado International Airport is the country's main air entry point.

\section{Challenges and Prospects}
Like any major metropolis, Bogotá faces significant challenges, including traffic management, pollution, security, and social inequalities. However, the city is also a dynamic center of innovation and development, with considerable potential for the future. Ongoing efforts aim to improve the quality of life for its inhabitants and strengthen its role as an influential South American capital.

\section{Conclusion}
Bogotá is a complex and fascinating city, offering a unique blend of colonial history, modern dynamism, and cultural richness. Its altitude, specific climate, and geographical location make it a singular Andean capital. Whether for its artistic heritage, its urban landscapes, or its vibrant atmosphere, Bogotá continues to attract and surprise its visitors.

Efforts are continuously made by local authorities to enhance urban mobility while simultaneously addressing environmental concerns through green initiatives. The city is also actively working on social programs aimed at reducing inequality and fostering community development, ensuring that progress benefits all segments of society."

This way it is so "easy to get lost".

Fixed?

In conclusion, this approach significantly simplifies the process and minimizes the risk of confusion.

\begin{table}[h!]
    \centering
    \begin{tabular}{|c|c|}
        \hline
        f & $2$ \\
        \hline
        $3$ & $4$ \\
        \hline
    \end{tabular}
    \caption{$5$}
    \label{tab:my_label}
\end{table}

\end{document}










