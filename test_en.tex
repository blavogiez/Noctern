\documentclass{mytex}
\policeprincipale{mathpazo}
\policesecondaire{mathpazo}



%-------------------- Informations sur le rapport ----------------------% Remplir les informations qui seront utilisées sur la page de garde et dans les en-têtes/pieds de page\titre{Développement}
\UE{Typographie}
\sujet{Projet de typographie avec \LaTeX}
\enseignant{Dr. LaTeX}
\eleves{Letudiant Joel \\ Letudiante Joelle}

%-------------------- Début du contenu du document ----------------------\begin{document}

% Création de la page de garde\fairepagedegarde

% Activation des en-têtes et pieds de page personnalisés pour le reste du document\fairemarges

% Création de la table des matières\tabledematieres


%======================================================================================\section{Introduction et Concepts de Base}
%======================================================================================This document aims to demonstrate the multiple features offered by the class\LaTeX{} personalized\texttt{rapport.cls}. Each section will explore a set of specific commands and environments to illustrate their use and visual rendering.\info{
	Ce document est auto-descriptif. Le code \LaTeX{} qui le génère est un exemple direct de l'utilisation de la classe. N'hésitez pas à consulter le fichier \texttt{.tex} pour voir comment chaque élément est implémenté.
}

We will start with basic elements such as lists and different levels of titles, before moving on to more complex topics. The correct use of French typographic quotes is done with the command\verb- I'm sorry.\enquote{...}Like this:\enquote{Ceci est un exemple}.\subsection{Listes à puces et numérotées}

The lists are stylized for better readability.\tsecnonum{Exemple de liste à puces (itemize)}
\begin{itemize}
	\item First item in the list.\item Second element, which can extend over several lines if necessary to demonstrate indentation.\item Third item with a sublist:\begin{itemize}
		\item Sub-element A.\item Sub-element B.\end{itemize}
\end{itemize}

\tsecnonum{Exemple de liste numérotée (enumerate)}
\begin{enumerate}
	\item The first step is always the most important.\item The second follows logically.\item And so on, with a clear and bold numbering.\end{enumerate}

\subsection{Structure des Titres}
The class defines a particular style for sections, sub-sections and sub-sub-sections, as you can see throughout this document. Here is an example of the hierarchy.\subsubsection{Ceci est un sous-sous-titre}
The style is more discreet to indicate a lower level of detail.%======================================================================================\section{Mathématiques et Environnements Scientifiques}
%======================================================================================The class incorporates powerful tools for writing scientific content, including mathematics and theorems.\subsection{Équations et Formules}

The equations are numbered by section. Here are Maxwell equations, a classic example using the environment\texttt{align}.\begin{align}
	\nabla \cdot \mathbf{E} &=\frac{\rho}{\varepsilon_0} \label{eq:maxwell1} \\
	\nabla \cdot \mathbf{B} &= 0\label{eq:maxwell2} \\
	\nabla \times \mathbf{E} &= -\frac{\partial \mathbf{B}}{\partial t} \label{eq:maxwell3} \\
	\nabla \times \mathbf{B} &=\mu_0\left(\mathbf{J} +\varepsilon_0\frac{\partial \mathbf{E}}{\partial t} \right)\label{eq:maxwell4}
\end{align}

\info{
	L'équation de Gauss (\ref{eq:maxwell1}) est fondamentale en électromagnétisme.
}

\subsection{Théorèmes, Définitions et Remarques}
Predefined environments allow the scientific discourse to be structured.\begin{definition}[Groupe]
	A group is a non-empty set$G$with a law of internal composition$\ast$which is associative, admits a neutral element and for which each element admits a symmetrical one.\end{definition}

\begin{theorem}[Théorème de Lagrange]
	If$H$is a subgroup of a finished group$G$, then the order$H$divides the order of$G$.\end{theorem}

\begin{exemple}
	All the relative integers$\mathbb{Z}$with the addition is a group.\end{exemple}

\begin{remarque}
	All these boxes share a consistent style for a pleasant reading.\end{remarque}

\subsection{Unités Scientifiques}
The package\texttt{siunitx} is configured for French.\tsec{Utilisation de \texttt{siunitx}}
Planck's constant\nomenclature{$h$}{Constante de Planck} is about\num{6.626e-34}. The speed of light\nomenclature{$c$}{Vitesse de la lumière dans le vide} in the void is$c = \SI{299792458}{\meter\per\second}$.%======================================================================================\section{Éléments Visuels : Figures et Tableaux}
%======================================================================================\subsection{Insertion de Figures}
Custom order\verb- I'm sorry.\insererfigure- Simplifies the addition of framed images.% Utilisation de la commande personnalisée\insererfigure{logos/logo.png}{4cm}{Ceci est le logo principal, inséré avec notre commande personnalisée \texttt{\textbackslash insererfigure}.}{logo_principal}

For more complex needs, such as sub-figures, standard environments always work.\begin{figure}[H]
	\centering
	\begin{subfigure}{0.45\textwidth}
		\centering
		\includegraphics[width=0.8\linewidth]{logos/logo.png}
		\caption{Première sous-figure.}
		\label{fig:sub1}
	\end{subfigure}
	\hfill % Espace entre les deux figures\begin{subfigure}{0.45\textwidth}
		\centering
		\includegraphics[width=0.8\linewidth]{logos/logo_ECL.jpg}
		\caption{Deuxième sous-figure.}
		\label{fig:sub2}
	\end{subfigure}
	\caption{Exemple de figure avec deux sous-figures utilisant le package \texttt{subcaption}.}
	\label{fig:sousfigures}
\end{figure}


\subsection{Création de Tableaux}
The tables are stylized with\texttt{booktabs} for a professional rendering.\begin{table}[H]
	\centering
	\caption{Comparaison des caractéristiques de différents langages.}
	\label{tab:langages}
	\begin{tabular}{l >{\raggedright\arraybackslash}p{4cm} c c}
		\toprule
		\textbf{Langage} &\textbf{Caractéristique principale} &\textbf{Typage} &\textbf{Année} \\
		\midrule
		Python & Simplicity & Readability & Dynamics & 1991\\
		Java &\enquote{Write once, run anywhere} & Static & 1995\\
		C++ & Performance & System Control & Static & 1985\\
		\rowcolor{lightgray!50} % Exemple de couleur de ligne\multirow{-4}{*}{\rotatebox{90}{\textbf{Populaires}}} & & &\\
		\bottomrule
	\end{tabular}
\end{table}


%======================================================================================\section{Boîtes d'Information et Listings de Code}
%======================================================================================\subsection{Boîtes d'Information}
Several types of coloured boxes are available to highlight some information.\res{
	C'est la fin de l'expérience. Le \textbf{résultat} est positif et confirme notre hypothèse de départ.
}

\comp{
	Par \textbf{comparaison}, l'approche A est 50\% plus rapide que l'approche B, mais consomme plus de mémoire.
}

\obs{
	Une \textbf{observation} importante : le système devient instable lorsque la température dépasse \SI{100}{\celsius}.
}

\warning{
	\textbf{Attention} : ne modifiez jamais les fichiers du noyau directement, au risque de corrompre le système.
}

\subsection{Listings de Code}
Environment\texttt{codeboxlang} allows to display code with a syntax coloring adapted to the language.\tsec{Exemple de code Python}
\begin{codeboxlang}{python}
	# Simple script to greet the world def says_hello(name): """ This function displays a greeting message. """" print(f"Hello, {name}!") if __name___ == "_main__": say_hello("World")\end{codeboxlang}

\tsec{Exemple de code Java}
\begin{codeboxlang}{java}
	// File: HelloWorld.java public class HelloWorld { /** * The program entry point. */ public static void main(String[] args) { System.out.println("Hello, World from Java!"); } }\end{codeboxlang}

\tsec{Exemple de requête SQL}
\begin{codeboxlang}{sql}
	-- Select active users SELECT user_id, user_name, registration_date FROM users WHERE is_active = 1 ORDER BY registration_date DESC;\end{codeboxlang}


%======================================================================================\section{Nomenclature et Références}
%======================================================================================\subsection{Nomenclature}
The terms defined in the text with\verb- I'm sorry.\nomenclature- They're gathered here.\printnomenclature

\subsection{Références et Hyperliens}
The package\texttt{hyperref} is configured for internal and external links.\begin{itemize}
	\item An internal link to the section on mathematics: see section\ref{sec:mathématiques-et-environnements-scientifiques}.\item An internal link to the logo figure: see figure\ref{fig:logo_principal}.\item An external link to the LaTeX project website:\url{https://www.latex-project.org/}.\end{itemize}


%======================================================================================\section{Conclusion}
%======================================================================================This document has successfully explored much of the functionality of the class\texttt{rapport.cls}From layout to mathematical typography, graphic elements and code extracts, this class provides a robust and aesthetic framework for the writing of professional and academic reports.\paragraphIt's obvious that the class\texttt{rapport.cls} is a powerful tool for writing professional and academic reports, offering a full range of features to ensure an aesthetic and structured layout.. In addition, its integration of mathematical elements allows to obtain well-formatted documents in this specific field.. Finally, the presence of code extracts, such as LaTeX listings or Python, provides an additional advance in terms of clarity and ease of understanding for a wide range of readers.\merci

\end{document}

























