\documentclass{mytex}
\usepackage{xcolor}
\usepackage{fontawesome5}
\usepackage{babel}
\usepackage{tcolorbox}
\usepackage{tikz}
\usetikzlibrary{positioning, arrows.meta}
\usepackage{pgf-umlcd}
\usepackage{array}
\usepackage{booktabs}
\usepackage{amsmath} % Pour les flèches
\usepackage{setspace}
\usepackage{hyperref}

\title{Rapport ECL - Template} %Titre du fichier

\begin{document}
	
	%----------- Informations du rapport ---------
	
	\titre{Analyse de données} %Titre du fichier .pdf
	\UE{SAÉ S2.04} %Nom de la UE
	\sujet{Statistiques} %Nom du sujet
	
	\eleves{Baptiste \textsc{Lavogiez}}
	
	\enseignant{Marie \textsc{Deletombe}}
	
	%----------- Initialisation -------------------
	
	\fairemarges %Afficher les marges
	\fairepagedegarde %Créer la page de garde
	\tabledematieres %Créer la table de matières
	
	%------------ Corps du rapport ----------------
	
	\section{Contexte} 
	
	\begin{codeboxlang}[title=Physique des médaillés par sport]{sql}
		CREATE VIEW BodyMedalStats AS (
		SELECT sport, COALESCE(AVG(height),0) AS avg_height, COALESCE(AVG(weight),0) AS avg_weight
		FROM Participations
		JOIN Events USING(eno)
		JOIN Disciplines USING(dno)
		WHERE medal IS NOT NULL
		GROUP BY sport
		);
	\end{codeboxlang}
	
	\res{\insererfigure{img/version4/bodymedal.png}{10cm}{Physique des médaillés par sport}{fig::v4::bodymedal}}
	
	\tsecnonum{Le physique des non médaillés}
	
	\tsecnonum{Les plus maigres favorisés}
	\begin{codeboxlang}[title=]{sql}
		SELECT sport, b1.avg_weight - b2.avg_weight AS weight_diff
		FROM BodyMedalStats b1
		JOIN BodyNoMedalStats b2 USING(sport) 
		WHERE b1.avg_weight - b2.avg_weight < 0
		ORDER BY weight_diff DESC;
	\end{codeboxlang}
	
	\merci
	
\end{document}






